\item {\bf burstdet\_timeclust}\\
{$[$}tt,chcnt,dtt{$]$} {$=$} BURSTDET\_TIMECLUST(spks, bin\_s, thr\_mean)
\\
Given a set of spikes SPKS loaded by LOADSPIKE or LOADSPIKE\_NOC, finds
synchronized bursts (on 5 or more channels).
\\
This works by first finding bursts on individual channels using TIMECLUST,
and then finding synchronized bursts by clustering the single-channel
bursts, again using TIMECLUST.
\\
Both BIN\_S and THR\_MEAN are optional; default values are 0.1 s and 5x
resp.

\item {\bf cleanctxt}\\
{$[$}ctxts, idx{$]$} {$=$} CLEANCTXT(contexts) returns cleaned up contexts:
\\
- The first and last 15 values are averaged and used to compute DC offset.
\\
- These two estimates are weighted according to their inverse variance.
\\
- The DC offset is subtracted.
\\
- If any sample in -1:-0.5 or 0.5:1.5 ms is more than half the peak at 0 ms,
 ~the spike is rejected.
\\
- Use CLEANCTXT(contexts, testidx, relthresh) to modify this test:
\\
\mbox{~}\mbox{~}\mbox{~}\mbox{~}TESTIDX are indices (1:74) of samples to test, 
\\
\mbox{~}\mbox{~}\mbox{~}\mbox{~}RELTHRESH is a number between 0 and 1.
\\
- Additionally, the area immediately surrounding the peak is tested at
 ~the 0.9 level: The spike is also rejected if any sample in -1:-0.16
 ~or 0.16:1.5 ms has an absolute value more than 0.9 x the absolute peak
 ~value. This test is modified on its outer edges by the edges of
 ~testidx, but cannot be modified independently.
\\
Returns: ctxts: the accepted contexts, with DC subtracted
\\
\mbox{~}\mbox{~}\mbox{~}\mbox{~}\mbox{~}\mbox{~}\mbox{~}\mbox{~}\mbox{~}idx: the index of accepted spikes.
\\
Requirements: contexts must be as read from loadspike, i.e. 74xN (or
\\
\mbox{~}\mbox{~}\mbox{~}\mbox{~}\mbox{~}\mbox{~}\mbox{~}\mbox{~}\mbox{~}75xN).
\\
Acknowledgment: The algorithm implemented by this function is due 
\\
\mbox{~}\mbox{~}\mbox{~}\mbox{~}\mbox{~}\mbox{~}\mbox{~}\mbox{~}\mbox{~}to Partha P Mitra.

\item {\bf cr2hw}\\
hw {$=$} CR2HW(c,r) converts row and column to hardware channel number.
c, r count from 1 to 8; hw counts from 0.
\\
hw {$=$} CR2HW(cr) converts combined row and column to hardware channel
number. cr can be in the range 11..88.
\\
Illegal c, r values return -1.
c, r or cr may be matrices, in which case the output is also a matrix.
\\
The dimensions of c, r must agree.

\item {\bf heightscat88}\\
HEIGHTSCAT88(spks) plots scatter plots of the (spontaneous activity) spikes
in SPKS. It stacks thin horizontal raster plots for each channel. Within
each plot, spikes are positioned based on their height (amplitude).
\\
SPKS must have been loaded using LOADSPIKE or LOADSPIKE\_NOC.
\\
Column-row numbers are computed from channel numbers using hw2cr.
\\
p {$=$} HEIGHTSCAT88(spks) returns the plot handle.

\item {\bf hist2dar}\\
mat{$=$}hist2dar(X,Y,nx,ny,flag) returns a matrix suitable for 
pcolor containing the crosstab counts of X and Y in automatically
selected bins: you pick the number of bins, the code determines the
edges based on the min and max values in X and Y.
\\
If optional argument FLAG is present, the result is also plotted.

\item {\bf hist2d}\\
mat{$=$}hist2d(X,Y,x0,dx,x1,y0,dy,y1,flag) returns a matrix suitable for 
pcolor containing the crosstab counts of X and Y in the bins edges defined by
x0:dx:x1 resp y0:dy:y1.
\\
If optional argument FLAG is present, the result is also plotted.
{$[$}mat,xx,yy{$]$}{$=$}hist2d(...) returns x and y arrays as well, so you can
call pcolor(xx,yy,mat).

\item {\bf hw2cr}\\
{$[$}c,r{$]$} {$=$} HW2CR(hw) converts the hardware channel number hw (0..63) to
column and row numbers (1..8).
\\
cr {$=$} HW2CR(hw) converts the hardware channel number hw (0..63) to
a 1+row+10*col format (11..88).
hw may be a matrix, in which case the result is also a matrix.
\\
Illegal hardware numbers result in c,r {$=$}{$=$} 0.

\item {\bf loaddesc}\\
d {$=$} LOADDESC(fn) reads the description file FN or FN.desc 
and returns a structure with values for each line read. 
\\
The fields are named from the label in the description file.
\\
All values are converted to double. Original strings are stored in
the field LABEL\_str.
\\
Repeated keys are stored in a cell array.

\item {\bf loadraw}\\
y{$=$}LOADRAW(fn) reads the raw MEA datafile fn and stores the result in y.
\\
y{$=$}LOADRAW(fn,range) reads the raw MEA datafile fn and converts the
digital values to voltages by multiplying by range/2048. 
\\
Range values 0,1,2,3 are interpreted specially:
\\
range value ~ electrode range (uV) ~ ~auxillary range (mV)
\\
\mbox{~}\mbox{~}\mbox{~}\mbox{~}\mbox{~}0 ~ ~ ~ ~ ~ ~ ~ 3410 ~ ~ ~ ~ ~ ~ ~ ~ 4092
\\
\mbox{~}\mbox{~}\mbox{~}\mbox{~}\mbox{~}1 ~ ~ ~ ~ ~ ~ ~ 1205 ~ ~ ~ ~ ~ ~ ~ ~ 1446
\\
\mbox{~}\mbox{~}\mbox{~}\mbox{~}\mbox{~}2 ~ ~ ~ ~ ~ ~ ~ ~683 ~ ~ ~ ~ ~ ~ ~ ~ ~819.6
\\
\mbox{~}\mbox{~}\mbox{~}\mbox{~}\mbox{~}3 ~ ~ ~ ~ ~ ~ ~ ~341 ~ ~ ~ ~ ~ ~ ~ ~ ~409.2
\\
"electrode range" is applied to channels 0..59, auxillary range is
applied to channels 60..63. Note that channel HW is stored in the
(HW+1)-th row of the output.

\item {\bf loadspike}\\
y{$=$}LOADSPIKE(fn) loads spikes from given filename into structure y
with members
 ~time ~ ~(1xN) (in samples)
 ~channel (1xN)
 ~height ~(1xN)
 ~width ~ (1xN)
 ~context (75xN)
 ~thresh ~(1xN)
\\
y{$=$}LOADSPIKE(fn,range,freq\_khz) converts times to seconds and width to
milliseconds using the specified frequency, and the height and
context data to microvolts by multiplying by RANGE/2048.
\\
As a special case, range{$=$}0..3 is interpreted as a MultiChannel Systems 
gain setting:
\\
range value ~ electrode range (uV) ~ ~auxillary range (mV)
\\
\mbox{~}\mbox{~}\mbox{~}\mbox{~}\mbox{~}0 ~ ~ ~ ~ ~ ~ ~ 3410 ~ ~ ~ ~ ~ ~ ~ ~ 4092
\\
\mbox{~}\mbox{~}\mbox{~}\mbox{~}\mbox{~}1 ~ ~ ~ ~ ~ ~ ~ 1205 ~ ~ ~ ~ ~ ~ ~ ~ 1446
\\
\mbox{~}\mbox{~}\mbox{~}\mbox{~}\mbox{~}2 ~ ~ ~ ~ ~ ~ ~ ~683 ~ ~ ~ ~ ~ ~ ~ ~ ~819.6
\\
\mbox{~}\mbox{~}\mbox{~}\mbox{~}\mbox{~}3 ~ ~ ~ ~ ~ ~ ~ ~341 ~ ~ ~ ~ ~ ~ ~ ~ ~409.2
\\
"electrode range" is applied to channels 0..59, auxillary range is
applied to channels 60..63.
\\
In this case, the frequency is set to 25 kHz unless specified.

\item {\bf loadspike\_noc}\\
y{$=$}LOADSPIKE\_NOC(fn) loads spikes from given filename into structure y
with members
 ~time ~ ~(1xN) (in samples)
 ~channel (1xN)
 ~height ~(1xN)
 ~width ~ (1xN)
 ~thresh ~(1xN)
\\
Context is not loaded.
\\
y{$=$}LOADSPIKE\_NOC(fn,range,freq\_khz) converts times to seconds and width to
milliseconds using the specified frequency, and the height and
context data to microvolts by multiplying by RANGE/2048.
\\
As a special case, range{$=$}0..3 is interpreted as a MultiChannel Systems
gain setting:
\\
range value ~ electrode range (uV) ~ ~auxillary range (mV)
\\
\mbox{~}\mbox{~}\mbox{~}\mbox{~}\mbox{~}0 ~ ~ ~ ~ ~ ~ ~ 3410 ~ ~ ~ ~ ~ ~ ~ ~ 4092
\\
\mbox{~}\mbox{~}\mbox{~}\mbox{~}\mbox{~}1 ~ ~ ~ ~ ~ ~ ~ 1205 ~ ~ ~ ~ ~ ~ ~ ~ 1446
\\
\mbox{~}\mbox{~}\mbox{~}\mbox{~}\mbox{~}2 ~ ~ ~ ~ ~ ~ ~ ~683 ~ ~ ~ ~ ~ ~ ~ ~ ~819.6
\\
\mbox{~}\mbox{~}\mbox{~}\mbox{~}\mbox{~}3 ~ ~ ~ ~ ~ ~ ~ ~341 ~ ~ ~ ~ ~ ~ ~ ~ ~409.2
\\
"electrode range" is applied to channels 0..59, auxillary range is
applied to channels 60..63.
\\
In this case, the frequency is set to 25 kHz unless specified.

\item {\bf randscat88}\\
RANDSCAT88(spks) plots scatter plots of the (spontaneous activity) spikes
in SPKS. It stacks thin horizontal raster plots for each channel. Within
each plot, spikes are randomly positioned vertically for clarity.
\\
SPKS must have been loaded using LOADSPIKE or LOADSPIKE\_NOC.
\\
Column-row numbers are computed from channel numbers using hw2cr.

\item {\bf timeclust}\\
{$[$}t0,cnt,dt{$]$} {$=$} TIMECLUST(tms\_s,bin\_s,thr\_mean,thr\_abs)
\\
Given a set of times TMS\_S and a bin size BIN\_S (both nominally in seconds),
find the locations, volumes, and widths of peaks in the time distribution.
\\
A peak is (primitively) defined as a contiguous area of bins exceeding
\\
THR\_ABS {\bf and} exceeding THR\_MEAN times the mean bin count. Either THR\_MEAN
or THR\_ABS may be left unspecified, in which case THR\_ABS defaults to {$>$}{$=$}2.

