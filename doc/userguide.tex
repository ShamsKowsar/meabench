% doc/userguide.tex: part of MEABench, an MEA recording and analysis tool
% Copyright (C) 2000-2006  Daniel Wagenaar (dwagenaar@ucsd.edu)
%
% This program is free software; you can redistribute it and/or modify
% it under the terms of the GNU General Public License as published by
% the Free Software Foundation; either version 2 of the License, or
% (at your option) any later version.
%
% This program is distributed in the hope that it will be useful,
% but WITHOUT ANY WARRANTY; without even the implied warranty of
% MERCHANTABILITY or FITNESS FOR A PARTICULAR PURPOSE.  See the
% GNU General Public License for more details.
%
% You should have received a copy of the GNU General Public License
% along with this program; if not, write to the Free Software
% Foundation, Inc., 59 Temple Place, Suite 330, Boston, MA  02111-1307  USA

\documentclass[12pt,oneside]{book}

%%\usepackage{ifpdf}
%%\ifpdf
%%	\def\gfxextn{.png}
%%\else
%%	\def\gfxextn{.eps}
%%\fi

\def\gfxextn{}


\usepackage[letterpaper,width=6.5in,height=9.5in]{geometry}

\usepackage{url}
\usepackage[english]{babel}
\usepackage{fancyhdr}
\usepackage{xspace}
\usepackage{epsfig}
\usepackage[perpage,multiple]{footmisc}
\pagestyle{fancy}
\raggedbottom
\renewcommand{\headrulewidth}{0.4pt}
\renewcommand{\footrulewidth}{0.0pt}
\fancyhf{}
\fancyhead[L]{Wagenaar, {\it MEABench}}
\fancyhead[C]{{\bf{\rightmark}}}
\fancyhead[R]{\thepage}
\renewcommand{\chaptermark}[1]{\markright{#1}}
\renewcommand{\sectionmark}[1]{}
\def\Meabench{{MEABench}\xspace}
\def\meabench{{MEABench}\xspace}
\def\prog#1{{\bf #1}}
\def\cmd#1{{\bf #1}}
\def\arg#1{{\it #1}}
\def\stream#1{{\it #1}\xspace}
\def\streamtype#1{{\sc #1}\xspace}
\def\raw{\streamtype{raw}}
\def\spike{\streamtype{spike}}
\def\rms{\streamtype{rms}}
\def\intr{{\bf Ctrl-C}\xspace}
\def\eoi{{\bf Ctrl-D}\xspace}
\def\filename#1{{\sf #1}\xspace}
\def\X#1{{$[\!\![$ \bf #1 $]\!\!]$}}
\def\cfig#1#2{\begin{center}\epsfig{file=#1\gfxextn,width=#2}\end{center}}

\begin{document}
\thispagestyle{empty}
\setcounter{page}{1}

~\vspace{.5in}

\begin{center}
\noindent{\LARGE\sc{MEABench}}

\vspace{10pt}

\noindent{\large{Daniel Wagenaar}}

\vspace{10pt}

\noindent{\large Release 1.2.5, March 2011}

\vspace{1in}

\noindent\begin{minipage}{.7\textwidth}
\noindent\bf
\Meabench is a set of command line and GUI utilities to process data
from the Multi Channel Systems MEA60 amplifier.  Adaptation for
different hardware should be straightforward.  \Meabench was designed with
extensibilitity in mind, and is fully modular. That means that filters
can be inserted anywhere in the data processing stream. This document
describes the basics of the toolset and the usage of the individual
programs.
\end{minipage}

\end{center}

\vfill%\vspace{2in}
{\scriptsize\bf Copyright \copyright{} Daniel Wagenaar 1999-2011.

{\Meabench} is free software. You can redistribute it 
and/or modify it under the terms of the GNU General Public License as
published by the Free Software Foundation; either version 2 of the
License, or (at your option) any later version. However, I encourage
you to contact me if you wish to do so.

    This program is distributed in the hope that it will be useful,
    but WITHOUT ANY WARRANTY; without even the implied warranty of
    MERCHANTABILITY or FITNESS FOR A PARTICULAR PURPOSE. See the GNU
    General Public License for more details.

    You should have received a copy of the GNU General Public License
    along with this program; if not, write to the Free Software
    Foundation, Inc., 59 Temple Place, Suite 330, Boston, MA
    02111-1307 USA.

You may redistribute and/or modify this documentation under the terms
of the GNU Public Documentation License as published by the Free
Software Foundation. However, as for the software, I encourage you to
contact me if you wish to do so.

The latest version of this document and of the software described in
it, is available for public download from
 \url{http://www.its.caltech.edu/~pinelab/wagenaar/meabench.html}.
}

\pagebreak

\tableofcontents

\pagebreak

\chapter{Introduction}

\Meabench is an open-source suite of programs for acquisition and
analysis of multi-electrode array (MEA) recordings. \Meabench was
developed by Daniel Wagenaar at Caltech, drawing on the excellent
example set by MultiChannel Systems' MC\_Rack 
suite\footnote{See \url{http://www.multichannelsystems.com}.}.

The software runs under Linux and other Unix variants, and is freely
distributable under the terms of the GNU Public License (see
\url{http://www.gnu.org/copyleft/gpl.html}). It offers the following
functionality:

\begin{itemize}
\item Acquisition of raw electrode data from MultiChannel Systems' MCCard;
\item Complete removal of mains (60 Hz) interference using template
    filtering;
\item Removal of stimulation artifacts using the SALPA 
  algorithm\footnote{D.\ A.\ Wagenaar, and S.\ M.\ Potter: Real-time
  multi-channel stimulus artifact suppression by local curve fitting. 
  \emph{J.\ Neurosci.\ Meth.} {\bf 120:2}, 2002, pp 113--120. This, and
  most other publications mentioned here, may be obtained from my
  website, \url{http://www.biology.ucsd.edu/~dwagenaa/pubs.html.}};
\item Online and offline detection of spikes;
\item Online visualization of electrode data and spikes;
\item Online sonification of spikes;
\item Continuous or windowed saving of raw data and spikes;
\item Saving of spike waveforms, for later spike sorting and analysis;
\item Replaying of raw and spike files, at any speed;
\item Instant-replay buffer for easy analysis of recent events;
\item Online generation of raster plots;
\item Continuous monitoring of varying noise levels;
\item A variety of utilities for analysis and data format conversion,
    including:
\begin{itemize}
     \item Averaging of electrode recordings over trials;
     \item Conversion of binary spike files to ASCII representation;
     \item Filtering of spike files based on any mathematical expression
        involving shape or timing parameters;
     \item Extraction of single channels from 64 channel streams;
     \item Splitting of long data files into trials;
     \item Splitting of long data files into channels;
     \item Computing spike rates;
     \item Detecting culture-wide bursts.
\end{itemize}
\item Matlab functions to import \meabench data\footnote{Users may
	also be interested in Uli Egert's
    	comprehensive set of matlab code for MEA data
    	analysis; freely available at
    	\url{http://www.brainworks.uni-freiburg.de/projects/mea/meatools/overview.htm}.};
\item A program to allow easy scripting of \meabench modules for
offline processing.
\end{itemize}
  
\Meabench is fully modular, and any user with some Unix programming
experience can extend it to fit her or his needs. Since \meabench can
stream live data to your extension modules, it is well suited, for
example, to drive real-time feedback systems. In fact, the ability to
communicate with other software or hardware in real-time was one of
the primary motives for the conception of \meabench. It allowed a
reliable, sub-100 ms feedback loop time in our Neurally Controlled
Animat\footnote{T.\ B.\ DeMarse, D.\ A.\ Wagenaar, A.\ W.\ Blau, and
S.\ M.\ Potter: The neurally controlled animat: Biological
brains acting with simulated bodies. \emph{Autonomous Robots} {\bf 11},
2001, pp 305--310.}\footnote{D.\ A.\ Wagenaar, and S.\ M.\ Potter: A
versatile all-channel stimulator for electrode arrays, with real-time
control. \emph{J. Neural Eng.} {\bf 1}, 2004, pp 39--44.}.

\Meabench was written primarily for use with the MultiChannel Systems
MEA hardware, and a driver is included for their MCCard data
acquisition card, written by Thomas B.\ DeMarse with advice from
MultiChannelSystems. If you use different data acquisition hardware,
you may still find \meabench useful, because, due to its modular
nature, it is possible to write plug-in modules to read data from your
hardware. An experimental driver for one such board (manufactured by
United Electronic Industries, but not endorsed by us at its
state of development as of Nov 2002) is included as well.

\Meabench has been in constant use in the Pine lab at
Caltech for over four years, and at Steve Potter's group at Georgia
Tech\footnote{Public website: \url{http://www.neuro.gatech.edu/groups/potter/index.html}} since its
beginning. It is also used by several other research groups in the US,
Europe and Asia.
\Meabench remains a work in progress; we welcome
suggestions for improvement (and bug reports). Please join in the
development by submitting your code (patches and improvements) for
inclusion in future releases.

Another introductory article about \meabench was presented at the IEEE
EMBS Conference on Neural Engineering\footnote{D.~A.~Wagenaar,
T.~B.~DeMarse, and S.~M.~Potter: MEABench: A toolset for
multi-electrode data acquisition and on-line analysis. \emph{2nd
International IEEE EMBS Conference on Neural Engineering}, Arlington,
VA, March 16--19, 2005.}.



\section{Conventions}

\def\Useri#1{{\tt #1}}
\def\Compo#1{{\sf #1}}
\def\Prompt#1{{\sf\bfseries #1}}
\def\Uprompt{\Prompt{\$}}
\def\Mprompt#1{\Prompt{#1$\mathbf >$}}
\def\useri#1{\Useri{#1}~\par}
\def\compo#1{\noindent\Compo{#1}~\par}
\def\prompt#1{\noindent\Prompt{#1}~}
\def\uprompt{\prompt{\$}}
\def\mprompt#1{\prompt{#1$\mathbf >$}}

Throughout this document, the names of \meabench programs are typeset
\prog{Like this}. When running \meabench programs, the capitalization
should be omitted. Commands defined within \meabench programs are set
\cmd{like this}. The names of \meabench streams look \stream{like
this}, and their types \streamtype{like this}. \intr means holding the
Ctrl key while pressing `C'. 

In examples, text you type is set
\Useri{like this}. Text the computer produces is set \Compo{like
this}, with prompts highlighted
\Prompt{like this}. The unix prompt is represented as \Uprompt.

In definitions of \meabench commands, parameter names are written \arg{like
this}, optional paramaters are enclosed in brackets [\arg{like this}],
and alternatives are separated by a vertical pipe: \arg{this}$|$\arg{that}.

\section{Acknowledgments}

I'd like to acknowledge valuable input and support from Steve Potter,
Tom DeMarse and Jerry Pine. We are all grateful for financial support
from the NIH-NINDS and the Burroughs-Wellcome Fund, and for cooperation,
technical support, and equipment from MultiChannel Systems. Portions of
\Meabench{} were written by Tom DeMarse, Ryan Haynes, John Rolston, and Brookes
Poo.

\chapter{Compilation and installation}\label{chap:install}

\def\unixprompt#1{\begin{quotation}\Uprompt~\Useri{#1}\end{quotation}}

If you intend to use \meabench with MultiChannel Systems hardware,
installation is very straightforward. The following is a step by step
guide.

\begin{itemize}
\item Make sure you have gcc 2.95 or later.
\item Make sure you have Qt 3.0 or later. (Qt 4 or later will \emph{not} work. Most linux distributions allow you to install Qt~3 and Qt~4 side by side.)
\item Make sure your kernel is 2.4.10 or later. (Kernel 2.6.x recommended.)
\item Download the latest version of \meabench, and unpack it:
  \unixprompt{tar xzf meabench-1.2.5.tar.gz}
\item Enter the directory:
  \unixprompt{cd meabench-1.2.5}
\item Choose the directory were you want to install \meabench, e.g.,
  \filename{/opt/meabench}, and configure:
  \unixprompt{./configure -{}-prefix=/opt/meabench -{}-with-hardware=mcs}
(The \filename{-{}-with-hardware} chooses the particular DAQ hardware
you are using. Currently defined values are \Useri{mcs} for MultiChannel
Systems' original MCCard, \Useri{mcsE} for ``revision E'' or later of
MCCard\footnote{If you don't know which version you have, first try
  \Useri{mcs}. If the data look good in \prog{scope}, you are set. If
  not, try \Useri{mcsE} instead.}, \Useri{uei} for United Electronic Industries'
PD2-MF64-14H cards, and \Useri{ni} for National Instruments
cards. These last two drivers are currently ``experimental''. I would
particularly like to hear your success stories or bug reports.)
\item Compile:
  \unixprompt{make}
This will take a while. You may see various compiler warnings, such
as: 
\begin{quotation}\Compo{SD\_BandFlt.C:97: warning: assignment to `short
int' from `float'}\end{quotation}
These can safely be ignored. Actual compiler {\it errors} are another story,
of course: If you see something like:
\begin{quotation}\Compo{
SD\_BandFlt.C:52: `sqr' undeclared (first use this function)\\
\mbox{}\kern\parindent make: *** [SD\_BandFlt.o] Error 1}\end{quotation}
you have discovered a bug, which I would like to hear about (see
\emph{Reporting bugs} in chapter~\ref{chap:help}.
\item If compilation went well, you may now complete the installation
by typing:
  \unixprompt{make install}
Depending on the installation location you have chosen using
\Useri{-{}-prefix=\ldots}, you may have to become super user (root)
before executing \Useri{make install}.
\item If the installation location you chose is in your {\it path}
already, you can run \meabench programs simply by typing their name. 
If not, you can either modify your path variable, or copy the program
\prog{mea} from the \meabench `bin' directory to a location in your
path. That will allow you to run \meabench programs by typing, e.g.,
\Useri{mea rawsrv}.
\item If you are using kernel 2.6.x,  \Useri{make install} will also have 
installed the MCCard driver in the \filename{/lib/modules} tree, so
you can load the driver simply by typing \Useri{modprobe MCCard}. 
(This probably requires super user privileges.)
\item If you are using kernel 2.4.x, you should manually copy  
\filename{MCCard-linux-2.4/MCCard.o} to an appropriate system location 
so that \Useri{modprobe} can load it. Alternatively, you can use
\Useri{insmod }.
\end{itemize}

You may have noticed that the installation procedure closely matches
the standard GNU style. Generic information about GNU style
installation is provided in the file \Compo{INSTALL} in the top
directory of the \meabench source tree.


\chapter{Examples}\label{chap:examples}

In this chapter you will learn how to do most common tasks with
\meabench; subsequent chapters will provide a reference guide to
individual components. I will assume that you have already compiled
\meabench and installed it in a place where your unix shell can find it.

\section{Ensuring that the driver is loaded}

[This assumes you are using the MultiChannel Systems driver; for other
drivers, contact the author of that specific driver.] Start by
checking that all hardware is connected properly. Then open a terminal
window, and type \Useri{lsmod}. If `MCCard' is listed in the resulting
output, you are good to go. Otherwise, try typing
\Useri{modprobe MCCard}. (You will likely have to run that as super
user; use \Useri{su} or \Useri{sudo} and be aware of the
responsibility associated with super user privileges.) If that gives
an error (e.g. the system cannot find the file), \Useri{cd} to the
directory where you built \meabench, then type \Useri{insmod
MCCard-linux-2.6/MCCard.ko} if you are using a 2.6.x kernel, or
\Useri{insmod MCCard-linux-2.4/MCCard.o} if you are using a 2.4.x
kernel. After all this, type \Useri{lsmod} again. This time, `MCCard'
should be listed. If not, contact the author.

\section{Displaying electrode traces on-line}

This example shows how to display incoming data fom an MEA in
real time. 

Start by checking that all hardware is connected properly, with an MEA
locked in the pre-amplifier. Then open two terminal windows.

\begin{itemize}
\item Run the visualization program, \prog{scope} from the first terminal:
\begin{quotation}
\uprompt~\useri{scope}
\compo{Error from Sockclient: Constructor failed at connect [Connection refused]}
\compo{Error from Sockclient: Constructor failed at connect [No such file or directory]}
\end{quotation}
The warnings are normal, and indicate that the scope is not yet
`connected' to any input. We will fix that in a moment. Let's first
take a look at the scope window:

\cfig{scope}{.8\textwidth}

The top area contains a lot of controls, which you will learn to use
gradually. The main part of the window displays an 8 by 8 grid of
rectangular panels, blank for now, which will show electrode traces in
a moment. The panels are layed out in the same shape as the physical
electrodes on an MCS MEA, with the corner positions occupying the
auxiliary channels A1 through A3. The bottom right corner is not
connected to anything on our data acquisition board. If you are using
a `hex' MEA, you can click the `Hex' button in the second row of
controls to change the layout appropriately.

\item Start the core data acquisition program, \prog{rawsrv} in the
second terminal:
\begin{quotation}
\uprompt~\useri{rawsrv}
\compo{This is rawsrv, compiled for use with the following hardware:}
\compo{}
\compo{Pre-amp: MEA1060 by MultiChannel Systems}
\compo{A/D:     MCCard by MultiChannel Systems}
\compo{Driver:  MCCard.o by Thomas DeMarse}
\compo{Plugin:  MCS by Daniel Wagenaar}
\compo{}
\mprompt{rawsrv}
\end{quotation}
The information printed by \prog{rawsrv} may be different if you configured
\meabench for different hardware.
\item Put the cursor in the `Raw source' box of \prog{scope}, and hit
return, or re-select \stream{raw} from the pull-down menu. The label
`Not found' should be replaced by `New/Ready', indicating that the
scope and \prog{rawsrv} are now connected. You will also notice that
the graphs now are adorned with a zero-line.
\item Return input focus to the terminal with \prog{rawsrv}, and set
the data acquisition running: 
\begin{quotation}
\mprompt{rawsrv}\useri{run}
\compo{Gain setting is 2 (+/- 0.683 mV full range)}
\compo{Trigger detection is disabled.}
\compo{Stimulation blankout is disabled.}
\compo{Stimulated channels: none}
\compo{Running...}
\end{quotation}
Notice that the time displayed at the far right of the second row of
controls in the scope runs.
\item Click the `Center' button of the scope to remove DC offsets from
the traces. Now may be the time to explore the other controls in the top
row of the scope. Help balloons pop up when you hover the mouse cursor
over any item.
\item To stop data acquisition, bring input focus to \prog{rawsrv},
and press \intr.
\begin{quotation}
\compo{(interrupt)}
\mprompt{rawsrv}
\end{quotation}
\item The scope may be terminated using the `Close' button provided by
your window manager. The \prog{rawsrv} should be terminated by
pressing \eoi, or by typing `quit':
\begin{quotation}
\mprompt{rawsrv}\useri{quit}
\uprompt
\end{quotation}
\end{itemize}

\section{Online spike detection}

The next example will extend the previous one, by adding a spike
detector, and a recorder to save the data to hard disk.
\begin{itemize}
\item Open four terminal windows, and start \prog{scope} and
\prog{rawsrv} as before. In the following, I will let you figure out
in which window to type from the prompts shown. For example, I'll
write
\begin{quotation}
\mprompt{rawsrv}\useri{run}
\end{quotation}
and leave it implicit that you need to bring keyboard focus to the
terminal window in which you started \prog{rawsrv}, and then type `run'.
\item Start the spike detector, and configure it:
\begin{quotation}
\uprompt\useri{spikedet}
\mprompt{spikedet}\useri{source raw}
\compo{Source is raw}
\mprompt{spikedet}\useri{type 3}
\compo{Type is BandFlt-25}
\mprompt{spikedet}\useri{thresh 5}
\compo{Threshold is 5}
\mprompt{spikedet}
\end{quotation}
The threshold is specified in terms of an estimate of RMS noise. 
The spike detector can only base this estimate on observing the noisy signal
and guessing which part is noise. This is not an exact science, and
\prog{spikedet} implements more than one algorithm. See section
\ref{ss:spikedet} for details. If you can think of a better scheme,
please contribute it!
\item Set the \prog{rawsrv} running:
\begin{quotation}
\mprompt{rawsrv}\useri{run}
\compo{Gain setting is 2 (+/- 0.683 mV full range)}
\compo{Trigger detection is disabled.}
\compo{Stimulation blankout is disabled.}
\compo{Stimulated channels: none}
\compo{Running...}
\end{quotation}
(Is the scope aware of the existence of \prog{rawsrv} and
\prog{spikedet}? If not, use the `Raw source' and `Spike source'
controls to remedy the situation.)
\item The spike detector needs to be `trained' on the amount of noise
in the source:
\begin{quotation}
\mprompt{spikedet}\useri{train}
\compo{Type is BandFlt-25}
\compo{Threshold is 5}
\compo{Training...}
\end{quotation}
\item A few seconds will pass, and then the spike detector will
return:
\begin{quotation}
\compo{Training complete}
\mprompt{spikedet}
\end{quotation}
\item Interrupt the \prog{rawsrv} as before.
\item Start and configure the recorder:
\begin{quotation}
\uprompt\useri{record}
\mprompt{record}\useri{cd}
\compo{Working directory: /home/wagenaar/tmp}
\mprompt{record}\useri{source raw spike}
\compo{Sources are:}
\compo{~~raw [raw]}
\compo{~~spike [spike]}
\end{quotation}
(The recorder tells you that it is about to record from \prog{rawsrv},
which provides a stream called \stream{raw} of type \raw, and from
\prog{spikesrv}, which provides a stream called \stream{spike} of type
\spike.)
\item Prepare the recording:
\begin{quotation}
\mprompt{record}\useri{record firstdata}
\compo{Sources are:}
\compo{~~raw [raw]}
\compo{~~spike [spike]}
\compo{Waiting for START from raw}
\compo{Waiting for START from spike}
\end{quotation}
The recorder will output files called `firstdata.raw', and
`firstdata.spike', but for now it's waiting for the data to arrive.
\item Start the spike detector:
\begin{quotation}
\mprompt{spikedet}\useri{run}
\compo{Source is raw}
\compo{Type is BandFlt-25}
\compo{Threshold is 5}
\compo{Quick recording disabled}
\compo{Excluded electrode channels are: none}
\compo{Disabled analog channels are: None}
\compo{Waiting for START from raw...}
\end{quotation}
As you can see, the spike detector is quite chatty, and tells you lots
of things you already know. It helps to prevent surprises later...
\item Let's record precisely 100 seconds:
\begin{quotation}
\mprompt{rawsrv}\useri{run 100}
\compo{Gain setting is 2 (+/- 0.683 mV full range)}
\compo{Trigger detection is disabled.}
\compo{Stimulation blankout is disabled.}
\compo{Stimulated channels: none}
\compo{Running...}
\end{quotation}
\item The spike detector will confirm that the run started:
\begin{quotation}
\compo{Running...}
\end{quotation}
As will the recorder:
\begin{quotation}
\compo{Recording from raw into firstdata.raw without triggering}
\compo{Recording from spike into firstdata.spike without triggering}
\end{quotation}
It is important to set the different programs in motion in the right
order: downstream first. The programs will automatically wait for
a START signal from their source, and they will wait forever if the
source is already running when they open the communication channel.
\item After 100 seconds, \prog{rawsrv} will stop, and the spike
detector will issue an end-of-run report:
\begin{quotation}
\compo{STOP received - 1511 spikes detected}
\compo{Buffer use percentages: 1}
\mprompt{spikedet}
\compo{(Client `record' lost)}
\mprompt{spikedet}
\end{quotation}
\item The recorder will also report some statistics:
\begin{quotation}
\compo{Recording from raw ended}
\compo{Buffer usage: 9}
\compo{Recording from spike ended}
\compo{Buffer usage: 1 4}
\mprompt{record}
\end{quotation}
Those `buffer usage' numbers are shown for each of the communication
channels between pairs of \meabench programs. If any number is above
80 (percent), the recorder will warn of the risk of buffer overruns.
\item Take a look at the `description' files that \prog{record}
generated. They are called `firstdata.raw.desc' and
`firstdata.spike.desc'. They contain more statistics about the run,
which may be helpful for your or your computer to interpret the data
later on.
\end{itemize}

\section{Triggered recording}

This final fully worked example explains how to perform
event-triggered recording. Let's assume that we have a stimulator
connected to the MEA, which delivers current pulses on one of the
channels once every few seconds. We want to record the responses.

\begin{itemize}
\item The
first step is to connect a TTL trigger from the stimulator to channel
A1 on the data acquisition card.
\item Then, set up the \prog{rawsrv} for trigger detection:
\begin{quotation}
\uprompt\useri{rawsrv}
\mprompt{rawsrv}\useri{trigchannel 1}
\compo{Trigger detection is disabled}
\mprompt{rawsrv}\useri{trigthresh 3000}
\compo{Trigger detection is disabled}
\mprompt{rawsrv}\useri{usetrig 1}
\compo{Trigger detection is enabled on channel A1 - threshold is 3000}
\end{quotation}
If you are using MCS hardware, zero volts is represented as digital
value 2048, and a TTL trigger will max out the amplifier at digital
value 4095, so a threshold halfway is appropriate. If you are using
different hardware, use the following trick: set the \prog{rawsrv}
running, and open a \prog{scope} on it. Control-double-click on the
panel displaying A1. In the controlling terminal, you will see a lot
of numbers scrolling past. Those are the digital values read from
channel A1. 50~ms of data is shown, one millisecond per line (if the
terminal window is wide enough). If you catch a trigger pulse using
the `Freeze' button, and display it in a 50~ms window, you can figure
out both the baseline value and the peak value, and set a threshold
based on those. Here's a part of the output I obtained doing this
experiment:
\begin{quotation}
\scriptsize\compo{~~36.746 2383 2384 2383 2383 2383 2383 3534 4095 4095 4095 4095 4095 4095 4095 4095 4095 4095 4095}
\compo{~~36.747 4095 4095 4095 4095 4095 4095 4095 4095 4095 4095 4095 4095 4095 4095 4095 2793 2446 2392}
\compo{~~36.748 2378 2371 2370 2373 2376 2379 2380 2381 2381 2381 2382 2382 2382 2382 2382 2381 2383 2382}
\end{quotation}
The trigger pulse is clearly visible as a sequence of maxed-out values
(4095).
\item Set up the recorder for triggered recording:
\begin{quotation}
\uprompt\useri{record}
\mprompt{record}\useri{source raw}
\compo{Sources are:}
\compo{~~raw [raw]}
\mprompt{record}\useri{trecord secondfile 50 450}
\compo{Sources are:}
\compo{~~raw [raw]}
\compo{Waiting for START from raw}
\end{quotation}
\item Start the data acquisition:
\begin{quotation}
\mprompt{rawsrv}\useri{run}
\compo{Running...}
\end{quotation}
\item Now set the stimulator going, and watch the progress using
\prog{scope}.
\item After the stimulation program has ended, stop the recording by
pressing \intr in \prog{rawsrv}'s terminal window. {\bf Do NOT
terminate the recorder, instead, terminate the source.} That way you are
guaranteed that the final trigger window is saved to disk properly.
\end{itemize}

\section{Using \prog{Commander}}

Here is an example script for commander that reads a \raw file from
`noisy.raw', filters 60~Hz noise out of it using a reference signal on
channel A2, and detects spikes using BandFlt at five times RMS noise.
The results are saved as `denoised.spike'.
\begin{quotation}\sf\small
\noindent  $\#$ Start programs
\\  new replay so:noisy.raw
\\  new filter60hz/60hz so:reraw lock:a2 nper:100
\\  new spikedet so:60hz ty:3
\\  new record so:spike
\\
\\  $\#$ Check whether they came up OK
\\  expect replay 1 replay$>$
\\  expect 60hz 1 60hz$>$
\\  expect spikedet 1 spikedet$>$
\\  expect replay 1 replay$>$
\\
\\  $\#$ Let's train the spike detector
\\  tell filter60hz cont
\\  tell replay run
\\  sleep 5
\\  tell spikedet train
\\  flush spikedet
\\  expect spikedet 20 spikedet$>$
\\  dieif spikedet STOP received before training complete
\\  flush replay
\\  intr replay
\\  expect replay 5 replay$>$
\\
\\  $\#$ Good, let's go.
\\  tell spikedet run
\\  tell record record denoised
\\  expect record 1 Waiting
\\  expect spikedet 1 Waiting
\\  tell replay run
\\  flush record
\\  expect record 1000 record$>$
\\
\\  $\#$ All done.
\\  quit
\end{quotation}

\chapter{Basics of the toolset}
 
\Meabench consists of a number of independent linux command line
programs. Some of these programs are \emph{servers}, that make the
result of their computations available to others. Others are simply
\emph{clients}, that read data from one (or several) of the servers,
but do not make their results available. Many programs are both client
and server. Such programs can be thought of as generic \emph{filters}.

This section describes some of the internals of \meabench,  essential
for potential developers, and hopefully helpful for users who want to
understand how things work. When first reading of this document, you
may wish to skip ahead to the next chapter, the list of components.

The core of \meabench is a library of C++ classes that can be used to
easily construct new components. This library is stored in the
\filename{meabench/base} subdirectory. Presently, the only library
documentation is contained in the source (header) files.

Components of \meabench communicate with each other in a standardized
way. Servers publish a \emph{shared memory stream}, from which clients
can read asynchronously. Currently, two data types are supported:
\raw, which contains raw digital data as read from the driver,
and \spike, which contains information about spikes.
Internally, \raw data is represented by C++ datatype
\emph{Sample}, while \spike data is represented as \emph{Spikeinfo}.
These datatypes are defined formally in \filename{meabench/common/Types.H}.

Associated with each stream is some header information, from which
clients can find out how much data is ready to be read from the stream.

Servers never check whether clients are keeping up --- it is up to the
individual clients to detect overruns. This philosophy was adopted
because some clients may not care too much about overruns (e.g.
display programs) whereas others do (filters and recorders). All
current core clients detect buffer overruns and report buffer usage at
the end of a data taking run.

In addition to shared memory streams, servers publish a \emph{wakeup
socket}, from which clients can receive wakeup calls whenever a given
amount of data is available in the stream. The wakeup socket also
notifies clients when a run starts or ends, and when triggers are
detected.





%%%%%%%%%%%%%%%%%%%%%%%%%%%%%%%%%%%%%%%%%%%%%%%%%%%%%%%%%%%%%%%%%%%%%%
%%%%%%%%%%%%%%%%%%%%%%%%%%%%%%%%%%%%%%%%%%%%%%%%%%%%%%%%%%%%%%%%%%%%%%
\renewcommand{\sectionmark}[1]{\markright{#1}}
\chapter{List of components}

These are the programs that currently make up \meabench. Following
sections detail each program.

\begin{itemize}
\item \prog{Rawsrv} --- The grandmother server. It reads data from the
hardware using Tom DeMarse's driver and makes it available as a
\raw stream. 

\item \prog{Spikedet} --- Basic spike detection. It reads from a
\raw stream, and publishes a \spike stream.

\item \prog{60hz} --- Template filter to reduce 60 Hz pickup.

\item \prog{Salpa} --- Stimulus artifact filter.

\item \prog{Record} --- Records \raw or \spike data to
disk.

\item \prog{Replay} --- Replays files created by \prog{Record}.

\item \prog{Scope} --- GUI program to display \raw and \spike data
online.

\item \prog{Spikesound} --- GUI program for online sonification of \spike data.

\item \prog{Flexraster} --- GUI program to display raster plots of \spike 
data online.

\item \prog{Monitor} --- A debugging aid, it shows the status
of all servers.

\item \prog{Neurosock} and \prog{Nssrv} --- An alternative to
\prog{Rawsrv} that allows one to dedicate one computer to data
acquisition, and another for online analysis.
\end{itemize}

The following is a set of utility programs that can be used with
\meabench or on their own right.

\begin{itemize}
\item \prog{Runmeab} --- Opens a set of xterms from which \meabench
programs can be launched.

\item \prog{Spikedump} --- Converts a \spike file into human readable form.

\item \prog{Doublectxt} --- Takes a \spike stream and a \raw stream,
and tags additional context on to the \spike information from the \raw
channel at which the spike occurs. 

\item \prog{Noisehisto} --- Takes a \raw stream and outputs a
histogram of voltages observed in each channel.

\item \prog{Noiseshape} --- Takes a \raw stream and outputs the first
few central moments of the voltage distribution for each channel.

\item \prog{Uniquespike} --- Output spikes found in one but not in
another file.

\item \prog{Trigvar} --- Computes the variance in a triggered \raw
stream as a function of time post stimulus.

\item A growing set of additional utilities needs to be documented.
\end{itemize}

Note that command names are usually capitalized in this
manual, but must always be spelled all lowercase on the command line.


\renewcommand{\sectionmark}[1]{\markright{#1}}

\chapter{Details of each component}

This section explains the core \meabench components in more detail.
Most of these components have a command line interface. Thus, the
following entries focus mostly on the available commands. In
addition to the commands listed in the individual descriptions below,
the following commands are common across components:

\begin{itemize}
\item \cmd{?} --- Provide help in the form of a list of commands with
brief descriptions.
\item \cmd{quit} --- Terminates the program gracefully. The same can be
effected by pressing \eoi at the prompt.
\end{itemize}

Servers that are capable of loading and saving data to disk  support:

\begin{itemize}
\item \cmd{cd} [\arg{directory}] --- Change or report current working
directory.

\item \cmd{ls} [\arg{arguments}] --- Directory listing as per /bin/ls.

\item \cmd{mkdir} [\arg{directory}] --- Create a new directory.

\item \cmd{!} \arg{command} [\arg{args}] --- Execute an arbitrary shell command.

\end{itemize}

Client programs support:

\begin{itemize}
\item \cmd{source} [\arg{stream-name}] --- Specify from which
other \meabench program the data is to be taken by specifying its
stream-name.
\end{itemize}


\section{Rawsrv}

\prog{Rawsrv} reads \raw data from the hardware and publishes it as a
shared memory stream called \stream{raw}. It is an extremely
straightforward piece of software. Other than providing a nice large
buffer to prevent overruns, it is able to detect trigger signals on
any of the analog channels (A1, A2 or A3) and to blank out the
electrode channels for some time during and after a trigger.

These commands are supported:

\begin{itemize}
\item \cmd{run} [\arg{time-in-s}] --- Starts a run. Optional argument limits
the duration of a run to the given time. Otherwise, press \intr to
stop a run.

\item \cmd{usetrig} [0/1] --- Enables (1) or disables (0) reporting
trigger events on the wakeup socket and in the auxilliary data of the
\raw stream. 

\item \cmd{trigchannel} [1/2/3] --- Selects which of the three analog
channels to monitor for triggers.

\item \cmd{trigthreshold} [\arg{digivalue}] --- Specifies the (digital)
value of the threshold above which a trigger is detected.

\item \cmd{autothresh} [\arg{multiplier}] --- Sets the threshold for
trigger detection at {\it multiplier} standard
deviations above the channel mean value.

\item \cmd{gain} [\arg{gain-step}] --- Sets the gain of the MCS amplifier.
Type \cmd{gain ?} to list possible values. For our Multi Channel Systems
card, the values are as follows:
\begin{center}
\begin{tabular}{ccc}
\hline\vspace{-5pt}\\
value & full range ($\mathrm{uV}$) & step ($\mu \mathrm{V}$) \medskip\\
0 & 3.410 & 1.665 \\
1 & 1.205 & 0.588 \\
2 & 0.683 & 0.333 \\
3 & 0.341 & 0.167 \medskip\\
\hline
\end{tabular}
\end{center}

\item \cmd{blankout} [\arg{period-in-ms} $|$ 0] --- Enables or disables
blanking of electrode channels during a trigger. Blanking is performed by
replacing the data by the average of four samples obtained just prior
to the stimulus. Blanking only works if trigger detection is enabled.
Signal blanking is largely outdated by \prog{Salpa}.

\end{itemize}

\section{Spikedet}\label{ss:spikedet}

\prog{Spikedet} detects spikes on a \raw data stream and publishes
them as a \spike stream called \stream{spike}. Several different types
of spike detection may be supported by \prog{Spikedet}. 
Currently, these are the fully supported spikedetectors:
\begin{enumerate}
\item{BandFlt} --- Straightforward threshold detector;
\item{AdaFlt} --- Threshold detector with adaptive threshold;
\item{LimAda} --- Better threshold detector with adaptive threshold;
\item{SNEO} --- Detector based on instantaneous energy in signal.
\end{enumerate}

\noindent BandFlt is thoroughly tested and quite stable. AdaFlt 
(7/12/01) seems to work well except that it's threshold varies during
bursts; LimAda solves this problem. SNEO has never worked as well as I
would hope. See below for details on each detector.

These commands are supported:

\begin{itemize}
\item \cmd{source} --- See general description.
\item \cmd{type} [\arg{detector-name}] --- Changes or reports the detector
being used. Use \cmd{type ?} to query supported detectors.
\item \cmd{run} --- Starts a single run. If the detector hasn't been
trained yet, the first few seconds of the input data are used for training.
\item \cmd{cont} --- Same as \cmd{run}, but restarts automatically after
receiving a STOP command from the server.
\item \cmd{train} --- Trains the detector on the current source.
Unusually, this command does not wait for a START message from the
source to begin operation. It is recommended to let the source run for
a few seconds before commencing training, to ensure that any
transients have died out.
\item \cmd{info} --- Reports the result of training.
\item \cmd{threshold} [\arg{value}] --- Sets or reports the threshold for
spike detection, in units particular to each detector. Thresholds are
automatically scaled for each channel.
\item \cmd{disableanalog} [\arg{channel} \ldots $|$ --] --- Disables spike
detection on the given set of analog channels. Typical use is to
exclude the 60 Hz reference signal from being recorded in spike files.
\item \cmd{excludechannels} [\arg{RC} \ldots $|$ --] --- Marks a set of
electrode channels as `dead'. Useful to prevent spurious spike detection on
clamped down or flaky channels.
\item \cmd{outputfilt} [0/1] --- Enables or disables the creation of a
shared memory stream called \stream{spraw}, on which the filtered raw data is
reported.
\item \cmd{savenoise} \arg{filenamebase} --- Saves the current training results
as `{\it filenamebase}.noise'. This data consists of estimated RMS
noise values, and can be used by \prog{Salpa} as well.
\item \cmd{loadnoise} \arg{filenamebase} --- Loads a previous set of
training results from `{\it filenamebase}.noise'. \prog{Salpa}
generated noise files may be loaded as well.
\item \cmd{alias} --- A fake command. Reminds the user of the
existence of the `--alias' flag (see below).
\end{itemize}

In some circumstances it may be useful to run more than one spike
detector simultaneously. To make that work, \prog{Spikedet} can be
made to publish its results on a differently named stream, by starting
it as `spikedet -alias \emph{streamname} $[$\emph{cmds}$]$'.


\subsection{BandFlt}

BandFlt passes the \raw data through a band pass filter. Currently a
first order filter with cutoffs at 150 Hz and 2.5 kHz are used, but
this may be changed in `spikedet/Filters.H'. It detects spikes if the
filtered stream exceeds a given multiple of the estimated RMS noise in
each individual channel. Useful threshold values are 4 to 6. It should
be noted that noise in \raw data is far from Gaussian, so future
versions may be changed to employ more relevant noise measures. In the
current implementation, the noise is estimated by a slightly unusual
method, which is intended to minimize the effect of spikes and
stimulus artifacts on the estimate. Three hundred 10 ms windows of
electrode data is read. For each of these windows the RMS value is
calculated. The results are sorted, and the final estimate of RMS
noise is taken to be the 25th percentile of the measurements. While I
recognize that this method finds an underestimate of the RMS noise,
this algorithm is much more useful than straightforward RMS
measurement, for the stability reasons mentioned above.

\subsection{AdaFlt}

AdaFlt uses the same initial band pass filter and also collects 128
windows of length 10~ms from the beginning of the recording.
From then on, it proceeds differently: it measures the minimum and
maximum values in each of those windows, and finds the 40th percentile
of both collections of extrema. The initial thresholds for upward and
downward spikes are based on the result. While running, it keeps
collecting minima and maxima in 10 ms windows, although it uses only
one in ten windows\footnote{In fact, for every window it collects
extrema for only six out of 60 electrode channels, to spread the
computational load.}. Whenever 128 windows have been collected, the
thresholds for that channel are updated.

For every detected spike, the ruling threshold at the time of
detection is written into the \spike stream.

Useful threshold values are 1.3 to 2 (multiples of the extrema).

\subsection{LimAda}

After band pass filtering as for BandFlt, LimAda splits the data
stream into 10~ms windows, and determines the 2nd and 30th percentiles
of the distribution of voltages found in each such window. Call these
voltages $V_{.02}$ and $V_{.30}$. (Note that both are usually
negative because of the filtering, which sets $V_{.50} \sim \left<V\right>
\sim 0$.) It then
performs two tests:
\begin{itemize}
\item Is the ratio of $V_{.02}$ over $V_{.30}$ less than 5?
\item Is the absolute value of $V_{.30}$ (significantly) non-zero?
\end{itemize}
The first test makes sure that there was no actual spike in the
window; the second test makes sure that the data in the window was not
blanked out (e.g. by \prog{Rawsrv} or \prog{Salpa}). If both tests are
passed, the windows is considered `clean', and $V_{.02}$ is used to
update the current noise threshold estimate. Spikes are detected
whenever the absolute value of the voltage exceeds the current
threshold, which is the output of passing the absolute values of
$V_{.02}$ from all `clean' windows through a low-pass filter with a
time constant of 100 windows (1 second if all are clean). This
algorithm adapts rapidly to changing noise situations, while not
desensitizing during bursts.

The threshold settings are normalized to estimated RMS noise, so
values of 4 to 6 are reasonable. As of August 26, 2004, this is my
favorite spike detector.

\subsection{SNEO}

SNEO also passes the \raw data through a
band pass filter, but then computes the instantaneous energy in each
electrode stream:
\[  E_c(t) = V'_c(t)^2 - V_c(t) V''_c(t). \]
This energy is smoothed over 5 samples and spikes are detected if it
exceeds a given multiple of the RMS value of the energy. Although Kim
and Kim (IEEE Biomed eng 47 (2000) 1406) report that SNEO works very
well at S/N as low as unity, I am less convinced.

Useful threshold values seem to be 5 to 20.

\section{60hz}

\prog{60hz} provides a template filter to reduce 60 Hz line pickup in
raw data and publishes the results as \stream{60hz}. It works best if
an external lock in signal is provided on one of the analog lines. If
such a signal is not available, fast adaptation should be chosen to
reduce the effects of gradual desynchronization. Templates are
collected for each electrode channel independently.

These commands are supported:

\begin{itemize}
\item \cmd{source} --- See general description.
\item \cmd{run} --- Starts a single run. At the start of the run, a
small amount of data is used to train the filter. During this period
no output is generated.
\item \cmd{cont} --- Same as \cmd{run}, but restarts automatically after
receiving a STOP command from the server.
\item \cmd{nperiods} [\arg{periods}] --- Sets the adaptation time of the
filter. Old contents are decayed by a factor $1/e$ after the given
number of periods. (A period is 16.67 ms\footnote{Note for non-US
users or programmers: the period is specified in units of the sample
period by the variable REALPERIOD in ``60hz/Defs\.H''. When changing
this variable, please be aware that
MCCard's sampling frequency, although very constant, is 
not exactly 25.000~kHz.}.)
\item \cmd{templsize} [\arg{size-of-template}] --- Number of data points to
use for each template. This value is rounded internally to a power of
two. Low values reduce the efficacy of the filter, but high values
require longer training times and may make the system less stable. The
default value, 128, should normally be adequate.
\item \cmd{wait} [0/1] --- Enables (1) or disables (0) waiting for a
START command from the server. Operation with waiting disabled is
poorly tested and may result in desynchronized and useless recordings.
Not recommended for normal use.
\item \cmd{lockin} [-- $|$ A\arg{n}] --- Enables or disables the
use of an external synchronization pulse on a given analog channel.
Rising edge on the specified channel will be used to synchronize the
filters to the physical 60 Hz signal.
\item \cmd{limit} [\arg{adaptation-period-in-seconds} or 0 for unlimited]
--- It may be desirable to stop adaptation altogether after a certain
amount of time. For example, if very strong signals are expected
occasionally on the electrode channels. Such signals might be picked
up by the template and cause echos at 16.67 ms intervals. In practice,
\cmd{blockonmark} provides a better solution for most cases.
\item \cmd{blockonmark} [-- $|$ An [\arg{block\_ms} [\arg{thresh\_digi}
[\arg{lookahead\_ms}]]]] --- If your data contains (stimulation) artifacts,
the adaptive filter tends to create echoes of those artifacts. This
command can be used to temporarily suspend the adaptation (but not the
filtering) during artifacts. When enabled, \prog{filter60hz} will
detect upward threshold crossings on the specified analog channel
(\arg{An}), and disable adaptation on all electrode channels for the
given period (\arg{block\_ms}). The threshold is specified in digital
units (\arg{thresh\_digi}). The final argument (\arg{lookahead\_ms})
can be used if the marker may occur (a fraction of) a millisecond
after the start of an artifact\footnote{Even if the marker is timed to
exactly coincide with the start of the artifact, \arg{lookahead\_ms} can be
used to provide a safety margin of a few samples.}.
\end{itemize}

The current version of \prog{60hz} does not support \cmd{qrec}.

For off-line usage, the command \prog{Posthoc60hz} will read from a file
and output to a unix pipe. It has the additional benefit of skipping
artifacts when training. (This requires some command line switches ---
try `posthoc60hz --help'.)

\section{Salpa}

\prog{Salpa} is the artifact suppression algorithm described in
Wagenaar and Potter, Real-time multi-channel stimulus artifact
suppresion by local curve fitting, \emph{J.\ Neurosci.\ Meth.} {\bf
120:2} (2002) pp 113--120\@. Please refer to that paper for functional
details. You may pick up a preprint from
\url{http://www.biology.ucsd.edu/~dwagenaa/pubs.html}.

\prog{Salpa} supports the following commands:

\begin{itemize}
\item \cmd{run} --- Starts a single run. At the start of the run, a
small amount of data is used to train the filter. During this period
no output is generated.
\item \cmd{cont} --- Same as \cmd{run}, but restarts automatically after
receiving a STOP command from the server.
\item \cmd{digithresh} [\arg{digital-threshold}] --- Sets the threshold for acceptable
asymmetry in digital units, or reports the current value. In the
paper, this `asymmetry' is referred to as the {\it deviation} 
$\cal{D}$.
\item \cmd{noisethresh} [\arg{threshold-in-units-of-RMS-noise}] ---  Sets the threshold for acceptable
asymmetry in units of the estimated RMS noise, or reports the current
value.
\item \cmd{halfwidth} [\arg{halfwidth-in-ms}] --- Sets the {\sc salpa}
filter halfwidth. Except for a factor $\tau_{\textrm\scriptsize
sample}$, this is the number $N$ in the paper.
\item \cmd{asymduration} [\arg{asymmetry-window-width-in-ms}] --- The window
over which the asymmetry is measured.  Except for a factor $\tau_{\textrm\scriptsize
sample}$, this is the number $\delta$ in the paper.
\item \cmd{blankduration} [\arg{blanking-duration-in-ms}] --- Determines how
much signal is blanked even after the asymmetry (deviation) test has
been passed successfully. This is a bit of a hack, which was not used
in the paper.
\item \cmd{lookaheadwindow} [\arg{look-ahead-window-in-ms}] --- The last few
samples before the signal pegs are probably not entirely
artifact-free. This command allows you to blank a little bit of data
just before the stimulus artifact onset.
\item \cmd{digirails} [\arg{digi-rail1} [\arg{digi-rail2}]] --- Specifies which
digital values constitute the rails of the ADC. Defaults 0 and 4095
are for MultiChannel Systems hardware with \prog{Rawsrv}.
\item \cmd{fixedperiod} [\arg{period-ms} \arg{delay-ms} [\arg{blank-ms}]] --- In some
cases artifacts may occur that do not quite peg the channel. If this
happens in a triggered recording, \prog{Salpa} can still work: just
specify the length of the trigger window (\emph{period-ms}), the
amount of time before the onset of the artifact in each window
(\emph{delay-ms}), and the duration of the fast part of the artifact
(\emph{blank-ms}). Typical values for \emph{blank-ms} would be one or
two milliseconds. \prog{Salpa} will kick in after that and determine
the end of the irrepairable part of the artifact using the asymmetry
(deviation) test as usual.
\item \cmd{pegontrigger} [-- $|$ A\arg{n} [\arg{blank-ms} [\arg{tresh-digi}]]] --- A more
flexible solution to the problem explained above. \prog{Salpa} can treat
a positive threshold crossing on one given channel (usually A1) as a
signal to consider all channels pegged. \emph{blank-ms} has the same
meaning as for \cmd{fixedperiod}, and \emph{thresh-digi} specifies the
threshold (in digital units) for the detection of stimulation markers.
\item \cmd{channels} [-- $|$ + $|$ \arg{CR} \ldots] --- Normally, \prog{salpa}
operates on all electrode channels. Using this command you can
restrict operation to any subset. This is useful when the artifacts on
most channels reliably  last less than 1 or 2~ms, so the \prog{salpa}
algorithm doesn't improve things. Say `channels CR1 CR2 \ldots' to limit
operation to channels CR1, CR2, \ldots; or `channels --' to operate on
all channels. Alternatively, say `channels + CR \ldots' to add
channels to an existing list, or `channels -- CR \ldots' to remove
channels from an existing list. `channels +' restores a list
previously removed by `channels --'. Channels excluded from operation
are still subject to blanking by \cmd{fixedperiod} or
\cmd{pegontrigger}. This is usually desirable. If not, you can set the
blanking period to zero, e.g. by `pegontrigger A1 0'.
\item \cmd{source} --- See general description.
\item \cmd{train} --- Estimate the RMS noise level of the input. This
takes a few seconds. \cmd{train} must be executed while the source is
already running, unlike \cmd{run} and \cmd{cont}, which wait for the
source to start.
\item \cmd{savenoise} --- See \prog{Spikedet}.
\item \cmd{loadnoise} --- See \prog{Spikedet}.
\item \cmd{info} --- See \prog{Spikedet}.
\end{itemize}
An off-line version of \prog{salpa} is available as well; it's called
\prog{posthocartifilt}, and supports most options of the \meabench
component through command line switches. The command is
self-documenting: type \Useri{posthocartifilt
-{}-help} for details.


\section{Record}

\prog{Record} is used to record \raw or \spike streams to disk.
The program can record several streams in parallel. \prog{Record} can
optionally respect the triggers on the associated wakeup socket. In
this case, only the parts of the stream immediately surrounding the
triggers are saved to disk, and an auxiliary file with trigger times
is constructed.

These commands are supported:

\begin{itemize}
\item \cmd{record} \arg{filename} [\arg{comments}] --- Starts recording to the
specified file. The filename is augmented by the type of the data.
Optional comments are saved to a description file, if enabled.
\item \cmd{multirecord} \arg{base-filename} [\arg{comments}] --- Starts recording
to many files one after the other. The filename is augmented by the
start time of each recording, and by the type of the data.
\item \cmd{trecord} \arg{filename pretrig-ms posttrig-ms} [\arg{comments}] --- As
\cmd{record}, but respects trigger information. The window of
recording is specified as time before the trigger and time after the
trigger, both in ms.
\item \cmd{multitrecord} \arg{base-filename pretrig-ms posttrig-ms}
[\arg{comments}] --- \cmd{multitrecord} is to \cmd{trecord} as
\cmd{multirecord} is to \cmd{record}.
\item \cmd{source} [\arg{name}[/\arg{type}] \ldots] --- Specifies the sources for
recording. \cmd{Record} knows the type of most core \meabench streams.
If it doesn't know for the stream you name, the type can be specified
by appending it to the stream name after a slash. (For example, if
\stream{coolsort} is  a new stream of type \spike, you would say
`source coolsort/spike'.) 
\item \cmd{describe} [0/1] --- Enables (1) or disables (0) the
generation of a description file (filename constructed by augmenting
the data filename by `.desc'). Description files contain lots of
useful information pertaining to a run and are in human readable form.
\end{itemize}

\noindent \prog{Record} can record from several sources simultaneously: just
specify them all together as arguments to \cmd{source}. For example,
`source raw 60hz spike' would prepare a recording from three sources.
If more than one stream of the same type is recorded, the streamnames
are incorporated in the filenames. Recording from several sources does
have a few limitations:
\begin{itemize}
\item Recording ends immediately when the first stream terminates. A
small amount of data from the end of the other streams may be lost.
\item For triggered recording, only the first stream will yield a
`.trig' file.
\end{itemize}
To avoid these limitations, it is possible to run several instances of
\prog{Record}, and have each of them record from a single stream. 

It is possible to terminate a recording session before the source ends
by pressing \intr in the \prog{Record} terminal window. This is
useful for recording a short segment of \raw data parallel to the
beginning of a longer \spike data recording.

\section{Replay}

\prog{Replay} replays files recorded by \prog{Record}. Currently,
\raw and \spike data are supported, and are published as
\stream{reraw} and \stream{respike} respectively. Replaying a \spike
file results in both a \raw and a \spike stream, the \raw stream
reporting the contexts stored in the \spike stream.

These commands are supported:

\begin{itemize}
\item \cmd{play} [\arg{filename} [\arg{type}]] --- Plays the given file. Normally,
\prog{Replay} automatically detects the type of the data. If it
doesn't, help it by specifying it explicitly. Without a filename,
plays the last played file again.
\item \cmd{slow} [\arg{slowdown-factor}] --- Slows down playback by a given
(real valued) factor. Useful for output to \prog{Scope}. Arguments
smaller than one cause speed up.
\item \cmd{run} --- Alias for \cmd{play} without arguments.
\item \cmd{source} [\arg{filename} [\arg{type}]] --- Specifies a filename (and
optional type) for later playback.
\item \cmd{blankout} [\arg{period-in-ms} or 0] --- Enables or disables
blanking of electrode channels during a trigger. Blanking is performed by
replacing the data by the average of four samples obtained just prior
to the stimulus. Blankout works only for triggered files.
\prog{Replay} does not detect stimuli itself.
\item \cmd{cd} [\arg{directory name}] --- Changes the current working directory.
\item \cmd{!} --- Shell escape. Useful, e.g. for `ls'.
\item \cmd{selftrig} [0/1] --- Enables or disables spike detection
from an analog channel in the stream. When disabled, triggers stored
in the \filename{.desc} file of a file produced by \prog{Record}'s
\cmd{trecord} command are still reported.
\item \cmd{trigchannel} [1/2/3] --- Sets the analog channel on which
triggers are detected (if \cmd{selftrig} is enabled).
\item \cmd{trigthreshold} [\arg{digivalue}] --- Sets the threshold for such
trigger detection.
\end{itemize}

\section{Scope}

\prog{Scope} shows raw data and spikes in a similar format as the
graphical parts of my older \prog{Qmeagraph} program, or Multi Channel
Systems' \prog{MCRack}. It mostly explains itself. Here are some
useful hints: 
\begin{itemize}
\item Double clicking on any of the small electrode traces opens a
separate window showing that channel.
\item Selecting window width, pre-trigger length, or voltage ranges in
\prog{Scope} does not affect recording in any way.
\item The scrollback buffer (enabled by clicking the `Freeze' button)
is very useful to home in on some interesting event. Raw data in the
scrollback buffer can be saved to disk using the `Save' button which
appears whenever `Freeze' is enabled. Currently, the scrollback buffer
is about 5 seconds long. This is a direct function of the length of
the shared memory segments used to pass \raw data around.
\item These known bugs exist in scope:
\begin{itemize}
\item Spike circles at the edges of the electrode traces leave
semi-permanent smudges.
\item When scrolling back, older spikes may lose their red marks if
there are many detected spikes. This is the result of the \spike data
shared memory segments being too short.
\item When replaying a spike stream, red spike marks sometimes appear
out of nowhere. These ghosts are easily recognized, because no context
data is plotted around them. This bug has been a mystery so far.
\end{itemize}
\end{itemize}

\section{Spikesound}

\prog{Spikesound} makes spikes audible through a PC sound card. It can
read of any \spike stream (e.g. straight from the spike detector, or
from \prog{replay}). The GUI controls are minimal:
\begin{itemize}
\item Source --- Selects the \meabench stream to get spikes from. That
usually is \stream{spike} to read from \prog{spikedet}, or
\stream{respike} to read from \prog{replay}.

\item Play --- Switch sound on or off

\item Volume --- Master volume control.

\item --ve only --- Limit sonification to downward spikes. The obvious
counterpart can be implemented on request.

\item Rethreshold --- It is often useful to set the spike detection
threshold fairly low in \prog{spikedet} and use off-line spike sorting
to clean up the data. However, hearing all the
near-noise-level spikes is not very appealing. This control allows you
to hear only strong spikes. For example, I like to set the
\prog{spikedet} threshold to $4.5 \sigma$, and set the
\prog{spikesound} rethreshold factor to $120\%$.

\item A1, A2, A3 --- Enable or disable beeps when a spike is detected
on one of the auxillary channels. For example, I like to use A1 and A3
for triggers, and A2 as a 60~Hz lock-in signal. So I may want to hear
a sound when a trigger happens, but I don't want to hear the 60~Hz
signal.
\end{itemize}

{\bf Customization note:} The current version uses output buffers of
about 25~ms to improve timing accuracy and reduce lag. If you prefer
hearing longer noises, or your sound card / CPU do not allow you to
use such short buffers, you may change the value of {\tt\small
AUDIO\_LOG\_FRAG} in {\tt\small spikesound/Audio.H}. The length of the
buffer is:
\[
	\tau = {2^{\,\textrm{\tt\scriptsize AUDIO\_LOG\_FRAG}}
	 \over 	176.4} \; {\mathrm{ms}}.
\]

\section{Flexraster}

\prog{Flexraster} displays raster plots of spike activity in triggered
recordings. It currently relies on `trigger spikes' on channel A1. 
There are six different ways to create rasters:
\begin{itemize}
\item Spont --- Spikes from all channels are combined as blue dots.
Each line of the raster is \emph{Pre} plus \emph{Post} ms wide. The
raster plot scrolls vertically, showing the most recent interval on
top. A new raster line is generated whenever there is a trigger on A1,
or every second if there has not been any trigger.
\item 8x8Rec --- Spikes from each channel are displayed in separate
panels. Within each panel, the display is as for Spont.
\item 8x8Stim --- If the shape of the trigger pulse is used to encode
a CR-position, each panel displays all spikes that occurred in
response to stimulation on a certain channel. See figure for details.
\item V Stim and H Stim --- Similar to 8x8Stim, but panels are
generated only for those CRs that actually receive stimuli.
\item Cont --- Arguably the most useful function, creates a scrollable
and scalable raster plot with 60 electrodes stacked vertically and time running
continuously left to right. Trigger pulses are indicated by red marks.
\end{itemize}
\prog{Flexraster} is under development. Suggestions for
improvement are especially welcome.

\section{Commander}

\prog{Commander} allows you to start and control \meabench 
programs from within a central shell-like language. This is mainly
useful for off-line analysis. You may find an example of its use in
chapter~\ref{chap:examples}. Commander logs all interaction with the
programs it controls to the screen, as well as to an optional logfile.
Creating logfiles is highly recommended, because it allows you to keep
track of how exactly you processed the data, and to check whether
\meabench behaved as you expected it to behave.

\prog{Commander} supports the following commands:
\begin{itemize}
\item \cmd{new} \arg{program}[/\arg{id}] [\arg{args}] --- Start a new
program. An optional \arg{id} may be assigned to the program to
disambiguate references to two instances of the same program, or for
ease of reference. \arg{Args} are passed to the program unchanged.
\item \cmd{tell} \arg{program|id command} [\arg{args}] --- Send a
command to the named program. \arg{Args} are passed unchanged.
\item \cmd{expect} \arg{program$|$id timeout regexp} --- Wait until
the named program produces output that matches \arg{regexp}. If this
does not happen within \arg{timeout} seconds, report an error.
\item \cmd{dieif} \arg{program$|$id regexp} --- Look back at the
output captured by the last expect, and abort if any line matches
\arg{regexp}. (This may be a line that was output long before the line
that made \cmd{expect} happy.)
\item \cmd{dieunless} \arg{program$|$id regexp} --- Look back at the
output captured by the last expect, and abort unless some line matches
\arg{regexp}. (This may be a line that was output long before the line
that made \cmd{expect} happy.)
\item \cmd{flush} \arg{program$|$id} --- Flush all output from the
named program, so that future \cmd{expect} commands will not match any
output prior to the \cmd{flush}.
\item \cmd{intr} \arg{program$|$id} --- Send an Interrupt signal to
the named program, i.e. simulate pressing \intr.
\item \cmd{kill} \arg{program$|$id} --- Send a Term signal to the
named program, terminating it. Normally, you'd use \cmd{close} to
achieve the same result more gracefully.
\item \cmd{close} \arg{program$|$id} --- Close a running program
normally, as if \eoi was pressed. If closing normally fails, the
program is terminated as per the \cmd{kill} command.
\item \cmd{wait} \arg{program$|$id} --- Wait for the named program to
terminate. Use this if you just sent it a \cmd{quit} command.
Normally, \cmd{close} is an easier way to terminate subprocesses.
\item \cmd{log} [\arg{logfile}] --- Write all future output to the
named file, or stop logging if no argument is given.
\item \cmd{comment} \arg{comments} --- Write the specified comments to
the log file.
\item \cmd{sleep} \arg{time\_s} --- Sleep for the given number of
seconds. The subprocesses are not affected.
\item \cmd{quit} --- Exit \prog{Commander} after closing all subprocesses.
\end{itemize}
Unlike other \meabench programs, commander does not present the user
with a prompt. You are not really expected to type away at commander
from a terminal (although you can). Instead, you'd normally prepare a
script, and then run commander on it:
\begin{quotation}
\uprompt~\useri{mea commander $<$ myscript.cmdr}
\end{quotation}
Any error that happens during script execution (e.g. failure to start
a subprocess or failure to read an \cmd{expect}-ed string) causes
\prog{Commander} to terminate immediately with an error report written
to the screen and the log file.

An auxillary program, \prog{Cmdlog2html} exists to convert log files
produced by \prog{Commander} to html format. Log files are human
readable, but the html format looks nicer.

\section{Monitor}

\prog{Monitor} is mostly a debugging tool. It displays the status of
each of the \meabench servers.

\section{Neurosock and NSsrv}

Run \prog{Neurosock} on the machine that contains the physical
hardware, and \meabench on any other machine that can connect to the
first machine through the internet. It can be run without any
arguments for 64-channel MCCards, or as \Useri{neurosock -set MC128}
for 128-channel MCCards. 

\prog{NSsrv} is exactly like \prog{Rawsrv}, except that it doesn't 
record directly from the MEA hardware. Instead, it connects to
\prog{Neurosock} on the same or another computer. For 64-channel
MCCards, run \prog{NSsrv} without arguments; to use the 2nd half of
128-channel cards, run it like \Useri{nssrv -s raw2}. 

\prog{NSsrv} has the same commands as \prog{Rawsrv}, with one
addition:
\begin{itemize}
\item \cmd{ip} aa.bb.cc.dd --- Specify the IP address of the computer
running \prog{Neurosock}.
\end{itemize}
Note: Only one instance of \prog{NSsrv} can read from a 64-channel
\prog{Neurosock} or half a 128-channel \prog{Neurosock} at a time. It
is not possible to specify gain separately for the two halves, and the
\cmd{gain} command will fail if you try to set gain on one half while
the other half is running.

\section{Spikedump}

\prog{Spikedump} converts \spike files to human readable form,
dropping the context data. It can read from a specified file, or
from stdin. It does not presently run of \meabench streams. Output are
time (in seconds), channel (hardware order, counting from zero),
height (digital value) and width (in samples).

\section{Doublectxt}

\prog{Doublectxt} combines a \raw stream with a \spike stream to
construct a file with two context fields per spike.
The resulting file can be read with
the \emph{loaddblctxt.m} matlab function.

\section{Other utilities}

This documentation is presently incomplete regarding the utilities in
the \filename{meabench/utils} and \filename{meabench/perl}
subdirectories. These utilities will provide a brief Usage message if
invoked with a `--\kern2pt--\kern2pt usage' argument.

\section{Matlab functions}

A number of matlab functions included with \meabench can be used to
load \meabench data files into matlab, to perform channel numbering
convention conversions and to perform some common visualization tasks.
These functions are installed in /opt/meabench/matlab (if you follow
the suggested installation procedure in chapter \ref{chap:install}).
In order to use them, you need to tell matlab about that directory:
\begin{quotation}
\Prompt{$>\kern-2pt>$}~\Useri{addpath('/opt/meabench/matlab');}
\end{quotation}

The functions are the following:

\begin{itemize}
\item {\bf burstdet\_timeclust}\\
{$[$}tt,chcnt,dtt{$]$} {$=$} BURSTDET\_TIMECLUST(spks, bin\_s, thr\_mean)
\\
Given a set of spikes SPKS loaded by LOADSPIKE or LOADSPIKE\_NOC, finds
synchronized bursts (on 5 or more channels).
\\
This works by first finding bursts on individual channels using TIMECLUST,
and then finding synchronized bursts by clustering the single-channel
bursts, again using TIMECLUST.
\\
Both BIN\_S and THR\_MEAN are optional; default values are 0.1 s and 5x
resp.

\item {\bf cleanctxt}\\
{$[$}ctxts, idx{$]$} {$=$} CLEANCTXT(contexts) returns cleaned up contexts:
\\
- The first and last 15 values are averaged and used to compute DC offset.
\\
- These two estimates are weighted according to their inverse variance.
\\
- The DC offset is subtracted.
\\
- If any sample in -1:-0.5 or 0.5:1.5 ms is more than half the peak at 0 ms,
 ~the spike is rejected.
\\
- Use CLEANCTXT(contexts, testidx, relthresh) to modify this test:
\\
\mbox{~}\mbox{~}\mbox{~}\mbox{~}TESTIDX are indices (1:74) of samples to test, 
\\
\mbox{~}\mbox{~}\mbox{~}\mbox{~}RELTHRESH is a number between 0 and 1.
\\
- Additionally, the area immediately surrounding the peak is tested at
 ~the 0.9 level: The spike is also rejected if any sample in -1:-0.16
 ~or 0.16:1.5 ms has an absolute value more than 0.9 x the absolute peak
 ~value. This test is modified on its outer edges by the edges of
 ~testidx, but cannot be modified independently.
\\
Returns: ctxts: the accepted contexts, with DC subtracted
\\
\mbox{~}\mbox{~}\mbox{~}\mbox{~}\mbox{~}\mbox{~}\mbox{~}\mbox{~}\mbox{~}idx: the index of accepted spikes.
\\
Requirements: contexts must be as read from loadspike, i.e. 74xN (or
\\
\mbox{~}\mbox{~}\mbox{~}\mbox{~}\mbox{~}\mbox{~}\mbox{~}\mbox{~}\mbox{~}75xN).
\\
Acknowledgment: The algorithm implemented by this function is due 
\\
\mbox{~}\mbox{~}\mbox{~}\mbox{~}\mbox{~}\mbox{~}\mbox{~}\mbox{~}\mbox{~}to Partha P Mitra.

\item {\bf cr2hw}\\
hw {$=$} CR2HW(c,r) converts row and column to hardware channel number.
c, r count from 1 to 8; hw counts from 0.
\\
hw {$=$} CR2HW(cr) converts combined row and column to hardware channel
number. cr can be in the range 11..88.
\\
Illegal c, r values return -1.
c, r or cr may be matrices, in which case the output is also a matrix.
\\
The dimensions of c, r must agree.

\item {\bf heightscat88}\\
HEIGHTSCAT88(spks) plots scatter plots of the (spontaneous activity) spikes
in SPKS. It stacks thin horizontal raster plots for each channel. Within
each plot, spikes are positioned based on their height (amplitude).
\\
SPKS must have been loaded using LOADSPIKE or LOADSPIKE\_NOC.
\\
Column-row numbers are computed from channel numbers using hw2cr.
\\
p {$=$} HEIGHTSCAT88(spks) returns the plot handle.

\item {\bf hist2dar}\\
mat{$=$}hist2dar(X,Y,nx,ny,flag) returns a matrix suitable for 
pcolor containing the crosstab counts of X and Y in automatically
selected bins: you pick the number of bins, the code determines the
edges based on the min and max values in X and Y.
\\
If optional argument FLAG is present, the result is also plotted.

\item {\bf hist2d}\\
mat{$=$}hist2d(X,Y,x0,dx,x1,y0,dy,y1,flag) returns a matrix suitable for 
pcolor containing the crosstab counts of X and Y in the bins edges defined by
x0:dx:x1 resp y0:dy:y1.
\\
If optional argument FLAG is present, the result is also plotted.
{$[$}mat,xx,yy{$]$}{$=$}hist2d(...) returns x and y arrays as well, so you can
call pcolor(xx,yy,mat).

\item {\bf hw2cr}\\
{$[$}c,r{$]$} {$=$} HW2CR(hw) converts the hardware channel number hw (0..63) to
column and row numbers (1..8).
\\
cr {$=$} HW2CR(hw) converts the hardware channel number hw (0..63) to
a 1+row+10*col format (11..88).
hw may be a matrix, in which case the result is also a matrix.
\\
Illegal hardware numbers result in c,r {$=$}{$=$} 0.

\item {\bf loaddesc}\\
d {$=$} LOADDESC(fn) reads the description file FN or FN.desc 
and returns a structure with values for each line read. 
\\
The fields are named from the label in the description file.
\\
All values are converted to double. Original strings are stored in
the field LABEL\_str.
\\
Repeated keys are stored in a cell array.

\item {\bf loadraw}\\
y{$=$}LOADRAW(fn) reads the raw MEA datafile fn and stores the result in y.
\\
y{$=$}LOADRAW(fn,range) reads the raw MEA datafile fn and converts the
digital values to voltages by multiplying by range/2048. 
\\
Range values 0,1,2,3 are interpreted specially:
\\
range value ~ electrode range (uV) ~ ~auxillary range (mV)
\\
\mbox{~}\mbox{~}\mbox{~}\mbox{~}\mbox{~}0 ~ ~ ~ ~ ~ ~ ~ 3410 ~ ~ ~ ~ ~ ~ ~ ~ 4092
\\
\mbox{~}\mbox{~}\mbox{~}\mbox{~}\mbox{~}1 ~ ~ ~ ~ ~ ~ ~ 1205 ~ ~ ~ ~ ~ ~ ~ ~ 1446
\\
\mbox{~}\mbox{~}\mbox{~}\mbox{~}\mbox{~}2 ~ ~ ~ ~ ~ ~ ~ ~683 ~ ~ ~ ~ ~ ~ ~ ~ ~819.6
\\
\mbox{~}\mbox{~}\mbox{~}\mbox{~}\mbox{~}3 ~ ~ ~ ~ ~ ~ ~ ~341 ~ ~ ~ ~ ~ ~ ~ ~ ~409.2
\\
"electrode range" is applied to channels 0..59, auxillary range is
applied to channels 60..63. Note that channel HW is stored in the
(HW+1)-th row of the output.

\item {\bf loadspike}\\
y{$=$}LOADSPIKE(fn) loads spikes from given filename into structure y
with members
 ~time ~ ~(1xN) (in samples)
 ~channel (1xN)
 ~height ~(1xN)
 ~width ~ (1xN)
 ~context (75xN)
 ~thresh ~(1xN)
\\
y{$=$}LOADSPIKE(fn,range,freq\_khz) converts times to seconds and width to
milliseconds using the specified frequency, and the height and
context data to microvolts by multiplying by RANGE/2048.
\\
As a special case, range{$=$}0..3 is interpreted as a MultiChannel Systems 
gain setting:
\\
range value ~ electrode range (uV) ~ ~auxillary range (mV)
\\
\mbox{~}\mbox{~}\mbox{~}\mbox{~}\mbox{~}0 ~ ~ ~ ~ ~ ~ ~ 3410 ~ ~ ~ ~ ~ ~ ~ ~ 4092
\\
\mbox{~}\mbox{~}\mbox{~}\mbox{~}\mbox{~}1 ~ ~ ~ ~ ~ ~ ~ 1205 ~ ~ ~ ~ ~ ~ ~ ~ 1446
\\
\mbox{~}\mbox{~}\mbox{~}\mbox{~}\mbox{~}2 ~ ~ ~ ~ ~ ~ ~ ~683 ~ ~ ~ ~ ~ ~ ~ ~ ~819.6
\\
\mbox{~}\mbox{~}\mbox{~}\mbox{~}\mbox{~}3 ~ ~ ~ ~ ~ ~ ~ ~341 ~ ~ ~ ~ ~ ~ ~ ~ ~409.2
\\
"electrode range" is applied to channels 0..59, auxillary range is
applied to channels 60..63.
\\
In this case, the frequency is set to 25 kHz unless specified.

\item {\bf loadspike\_noc}\\
y{$=$}LOADSPIKE\_NOC(fn) loads spikes from given filename into structure y
with members
 ~time ~ ~(1xN) (in samples)
 ~channel (1xN)
 ~height ~(1xN)
 ~width ~ (1xN)
 ~thresh ~(1xN)
\\
Context is not loaded.
\\
y{$=$}LOADSPIKE\_NOC(fn,range,freq\_khz) converts times to seconds and width to
milliseconds using the specified frequency, and the height and
context data to microvolts by multiplying by RANGE/2048.
\\
As a special case, range{$=$}0..3 is interpreted as a MultiChannel Systems
gain setting:
\\
range value ~ electrode range (uV) ~ ~auxillary range (mV)
\\
\mbox{~}\mbox{~}\mbox{~}\mbox{~}\mbox{~}0 ~ ~ ~ ~ ~ ~ ~ 3410 ~ ~ ~ ~ ~ ~ ~ ~ 4092
\\
\mbox{~}\mbox{~}\mbox{~}\mbox{~}\mbox{~}1 ~ ~ ~ ~ ~ ~ ~ 1205 ~ ~ ~ ~ ~ ~ ~ ~ 1446
\\
\mbox{~}\mbox{~}\mbox{~}\mbox{~}\mbox{~}2 ~ ~ ~ ~ ~ ~ ~ ~683 ~ ~ ~ ~ ~ ~ ~ ~ ~819.6
\\
\mbox{~}\mbox{~}\mbox{~}\mbox{~}\mbox{~}3 ~ ~ ~ ~ ~ ~ ~ ~341 ~ ~ ~ ~ ~ ~ ~ ~ ~409.2
\\
"electrode range" is applied to channels 0..59, auxillary range is
applied to channels 60..63.
\\
In this case, the frequency is set to 25 kHz unless specified.

\item {\bf randscat88}\\
RANDSCAT88(spks) plots scatter plots of the (spontaneous activity) spikes
in SPKS. It stacks thin horizontal raster plots for each channel. Within
each plot, spikes are randomly positioned vertically for clarity.
\\
SPKS must have been loaded using LOADSPIKE or LOADSPIKE\_NOC.
\\
Column-row numbers are computed from channel numbers using hw2cr.

\item {\bf timeclust}\\
{$[$}t0,cnt,dt{$]$} {$=$} TIMECLUST(tms\_s,bin\_s,thr\_mean,thr\_abs)
\\
Given a set of times TMS\_S and a bin size BIN\_S (both nominally in seconds),
find the locations, volumes, and widths of peaks in the time distribution.
\\
A peak is (primitively) defined as a contiguous area of bins exceeding
\\
THR\_ABS {\bf and} exceeding THR\_MEAN times the mean bin count. Either THR\_MEAN
or THR\_ABS may be left unspecified, in which case THR\_ABS defaults to {$>$}{$=$}2.


\end{itemize}

\chapter{File formats}

The current version of \meabench uses two binary file formats: \raw,
and \spike. All other files are plain text.

\section{Raw files}

Raw files contain electrode voltage values in digitized form. The
scaling factor between digital values and microvolts is not stored in
the \Compo{.raw} file, but in an accompanying \Compo{.raw.desc} file.
The number of channels in a raw file is currently fixed to 64, and not
noted in the \Compo{.raw.desc} file. This will be changed in a future
release. File format is: one 16-bit integer for every channel,
64~channels per scan, repeated for the length of the file. Thus, in (almost) C
notation, a raw file would be defined as:
\begin{quotation}
\compo{typedef short int scan[64];}
\compo{typedef scan rawfile[];}
\end{quotation}

\section{Spike files}

Spike files contain information about detected spikes. The time of
detection (in samples) as well as the channel number, height (in
digital units) and width (in samples) are stored in a structure that
also contains a limited amount of `context', or sample values
immediately surrounding the spike. The BandFlt and AdaFlt spike
detectors also record the threshold used for detecting that particular
spike in the structure. In (almost) C notation, a spike file would be
defined as:
\begin{quotation}
\compo{typedef struct \{}
\compo{ ~ long long time; // 64 bits of time (in sample periods from
start o ~ f file)}
\compo{ ~ short channel; // a channel number, 0--63}
\compo{ ~ short height; // height from baseline, possibly negative (in digital units)}
\compo{ ~ short width; // the width of the spike (in samples)}
\compo{ ~ short context[74]; // 24 samples before and 49 samples after peak}
\compo{ ~ short threshold; // detection threshold used for this spike}
\compo{\} spike;}
\compo{typedef spike spikefile[];}
\end{quotation}
As for raw files, conversion factors may be found in a description
file (\Compo{.spike.desc}).

%% \section{RMS files}
%% 
%% RMS files contain root-mean-square noise estimates from raw files. The
%% file format is:
%% \begin{quotation}\sf
%% \compo{typedef float rmsscan[64];}
%% \compo{typedef rmsscan rmsfile[];}
%% \end{quotation}
%% 
%% The values in \Compo{rmsscan} are in digital units, but stored in a
%% \Compo{float} rather than a \Compo{short int} for better precision.
%% 

\renewcommand{\sectionmark}[1]{}

\chapter{Hints and tips}\label{chap:help}

This section contains solutions to some common problems and provides
some hopefully helpful hints.


\section{Help! None of the programs will run.}

You may find that none of the programs will run, and complain like
this:
\begin{quotation}
\compo{/opt/meabench/bin/replay: error in loading shared libraries: libmea.so:
cannot open shared object file: No such file or directory}
\end{quotation}
This means that they cannot find the common libraries \emph{libmea.so}
or \emph{libmeagui.so}. This problem may be fixed by typing
\begin{quotation}
\noindent\verb+export LD_LIBRARY_PATH=/opt/meabench/lib:$LD_LIBRARY_PATH+ %$
\end{quotation}
before running the command. Similarly, if the perl programs complain about missing libraries, try:
\begin{quotation}
\noindent\verb+export PERL5LIB=/opt/meabench/libexec:$PERL5LIB+%$
\end{quotation}
(If you installed \meabench somewhere else, you'll know how to
modify the above lines.)

If you use \prog{mea}
this problem is less likely to show.

\section{Shared memory problems}

You may find that a server program refuses to run, complaining like
this:
\begin{quotation}
\compo{Error from ShmSrv: Segment exists, please delete using 
`ipcrm shm 3417399'}
\end{quotation}
This normally means that the server crashed on a previous
run, leaving behind
the shared memory segment used for its output stream. Executing the
suggested command often solves the problem, but you may be rewarded by
this when you rerun the program:
\begin{quotation}
\compo{Error from ShmSrv: Segment exists, and cannot be accessed. Any
lingering clients? Before deleting the segment using ipcrm (see
manpage) these may have to be stopped.}
\end{quotation}
This means that some client is still connected to the (defunct)
crashed server. Quitting and restarting such clients is a pretty sure
way of solving the problem. If quitting the client is undesirable,
most clients can be convinced to renegotiate the connection to their
server using the command \cmd{source}. Once the clients have been
cleared, the segment normally goes away spontaneously. If the problem
persists, you may have to resort to \prog{ipcs} and \prog{ipcrm}. See
the linux man pages for details.
\begin{quotation}
\compo{Error from ShmSrv: creat}
\end{quotation}
This exceedingly unhelpful error message may mean that you
don't have a \filename{.meabench} subdirectory in your home directory. 
This is where \meabench tries to link all its shared memory segments
and wakeup sockets. To make this problem go away for ever, create the
directory using \verb!mkdir ~/.meabench!.

\section{Help! Client X keeps saying `waiting for START from Y' and won't
run.}

Most clients synchronize the start of their run with the start of a
run of the server they are connected to. To make this work smoothly,
start the client off (typically by typing \cmd{run}) \emph{before} starting
the server. Some clients allow you to disable this synchronization
behavior, e.g. through a \cmd{wait} command.

\section{Abbreviating commands}

All programs that provide the user with a command line accept
non-ambiguous abbreviations for commands. For example, in most
programs, \cmd{run} can be abbreviated to \cmd{r}.

\section{Passing commands at run time}

All programs that provide the user with a command line can also
process commands passed at startup time. For example, instead of
entering into the following dialog:
\begin{quotation}
\uprompt\useri{spikedet}
\mprompt{spikedet}\useri{type 3}
\compo{Type is BandFlt-25}
\mprompt{spikedet}\useri{threshold 4}
\compo{Threshold is 4}
\mprompt{spikedet}\useri{source 60hz}
\compo{Source is 60hz}
\mprompt{spikedet}
\end{quotation}
you may also say:
\begin{quotation}
\uprompt\useri{spikedet ty:3 thr:4 so:60hz}
\compo{Type is BandFlt-25}
\compo{Threshold is 4}
\compo{Source is 60hz}
\mprompt{spikedet}
\end{quotation}
This is especially useful if you find yourself quitting and restarting
a program frequently, since with the second scheme, the shell history
facilities can be used.

At start up time, commands are separated from their arguments by a
colon rather than a space, and arguments are separated from each other
by a comma rather than a space.

\section{Interrupting long commands}

All command line programs respond to \intr by returning to
their prompt. If they don't respond immediately, press \intr
again after a second or two. Pressing \intr twice in quick
succession kills the program forcefully, quite possibly leaving a
shared memory segment behind as explained above. Pressing \intr
while the program is waiting for user input no longer exits the
program. All programs support a
\cmd{quit} command to exit cleanly, but \eoi (end of input) is
also an acceptable way to exit them.

\section{Debugging information}

Most programs support two additional commands to aid debugging. These
are: 

\begin{itemize}
\item \cmd{dbx} [0/1] --- Enables (1) or disables (0) debugging
output. Debugging output varies wildly from component to component and
from release to release. Mainly of use for developers, who can
sprinkle their code with sdbx(\ldots) calls to track bugs down.

\item \cmd{clients} --- Prints a list of all clients currently
connected to this server program. Mainly intended for debugging, the
output format is not very user friendly.
\end{itemize}


\section{Contacting the author}

For further information, or to report bugs, please contact:

\begin{quotation}
\noindent Daniel Wagenaar\\
UC San Diego, Neurobiology Section
9500 Gilman Drive m/c 0357\\
La Jolla, CA 92093\\
{\bf dwagenaar@ucsd.edu}
\end{quotation}

\noindent Suggestion for improvement are always welcome, but I cannot
guarantee I'll have time to implement them. You too can contribute
by sending me your bug fixing and feature adding patches by e-mail.
Please use `diff --\kern2pt C2' against the latest public release version.

The latest versions of \meabench and this documentation can be found on

\begin{center}\noindent\url{http://www.its.caltech.edu/~pinelab/wagenaar/meabench.html}\end{center}
\section{Reporting bugs}

When reporting bugs, please include the following information:
\begin{itemize}
\item The version of \meabench you are using.
\item The version of g++ you are using (type \Useri{g++ -{}-version}
to get this information).
\item The version of Qt you are using (type \Useri{moc -v} to get this
information).
\item The output of the `configure' script.
\item If compilation failed: the output of the `make' command (not
just the lines containing the error).
\item For runtime errors: Which programs you are running, which
commands you have executed within those programs, and a description of
what you are trying to do as well as any relevant screen output. A
succinct set of conditions that reproducibly produces the error is
very much appreciated.
\end{itemize}


\end{document}
